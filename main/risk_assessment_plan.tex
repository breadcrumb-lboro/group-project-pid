\section{Risk Management}

This section identifies any risks associated with the project, determines the likelihood of their occurrence and hypothesis their overall impact on the project. Furthermore, an attempt is made to categorise each as either technical or non-technical risk.~\parencite{dawson} asserts that technical risks refer to any risk associated with the software development process, whereas non-technical risks are associated with the project management process.

\begin{table}
  \caption[Project Risk Assessment]{Project Risk Assessment \hfill}
  \label{tab:runtime}
  \begin{tabular*}{14.0cm}{p{4.0cm}p{5.0cm}p{1.5cm}p{2cm}}
    \toprule
    Risk Item & Risk Management Technique & Risk Impact & Risk Category \\
    \midrule
    Group loses sense of urgency lost during earlier stages. It could be because the project is not clearly understood yet. & Group must meet regularly and report back to one another on task progress to generate a sense of accountability. & $1 \times 5 = 5$ & Non-Technical \\
    \addlinespace[0.5em]
    As the project progresses, scope creep begins to threaten deadlines. The group over-promises but under-delivers. & No new item can be added to the product backlog or current sprint unless it is deemed critical to the project’s success. & $2 \times 4 = 8$ & Technical \\
    \addlinespace[0.5em]
    The solution proposed by the group does not meet the end user’s need as the business model is fundamentally flawed. & Group must thoroughly analyse and gather solution requirements during the early stages of the project. & $2 \times 5 = 10$ & Non-Technical \\
    \addlinespace[0.5em]
    Project files (reports, databases or code) are corrupted or lost, as a result the group needs to recreate the affected files. & Group must use a version control system to track code changes and backup documents on a cloud storage service. & $2 \times 5 = 10$ & Technical \\
    \addlinespace[0.5em]
    Group realises it lacks the technical skills required to develop certain features in the solution, so they are omitted. & Group must give itself enough time at the start of the project to properly understand the development environment. & $1 \times 5 = 5$ & Technical \\
    \addlinespace[0.5em]
    Group regularly moves onto the next sprint without finishing all items picked up in one or more previous sprints. & Group must ensure that they have enough time to complete assigned items per sprint and not over-promise anything. & $1 \times 5 = 5$ & Non-Technical \\
    \addlinespace[0.5em]
    Group does not meet regularly, and due to a lack of communication it is hard to know what everyone is doing. & Group must endeavour to consistently adopt working practices that promote transparency and accountability. & $2 \times 3 = 6$ & Non-Technical \\
    \addlinespace[0.5em]
    Group member accidently overrides another member’s work because they didn’t have the latest version of the code. & Group must use a version control system to track code changes and backup documents on a cloud storage service. & $1 \times 3 = 3$ & Technical \\
    \addlinespace[0.5em]
    Group spends too much time discussing what needs to be done, and consequently has no time to work on solution. & Group leader must have an agenda prepared prior to every meeting and draft a project plan that has everyone’s support. & $2 \times 4 = 8$ & Non-Technical \\
    \addlinespace[0.5em]
    Group does not make it to the competition’s next round. & Make adjustments if necessary and continue development. & $2 \times 0 = 0$ & Non-Technical \\
    \bottomrule
  \end{tabular*}
\end{table}
