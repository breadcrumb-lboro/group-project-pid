\section{Conclusion}

The purpose of this document was to define the solution, act as a pit-stop for new ideas and set the benchmark for future success. As a result, the group can now easily determine:

\begin{itemize}
  \item What the solution is aiming to achieve
  \item Why it is important to achieve the stated aims and objectives
  \item How quality will be maintained throughout the course of the project
  \item Which role and responsibility each team member has been assigned
  \item How and when the arrangements discussed in this document will be put into effect
\end{itemize}

The group spent a considerable amount of time interacting with stakeholders in order to determine whether each idea presented was viable. Although these conversations were constructive, and from this the group determined that there are still areas of improvements for the group to take into consideration moving forward. 

\begin{figure}
  \centering
  \begin{minipage}{10cm}
    \centering
    \includegraphics[width=10cm]{inc/duolingo_gamification.jpg}
    \caption{Duolingo's Gamification Feature}
    \label{fig:duolingo_gamification}
  \end{minipage}
\end{figure}

For instance, not much thought has been put into the gamification element of the app, which was initially seen as a means of \emph{``motivating the user to keep returning due to their perceived fairness of the transaction''}, especially to increase repeat, regular usage amongst users on the Personal offering. Gamification is an umbrella term describing the use of game elements in non-gaming systems to improve user experience and user engagement~\parencite{deterding2011gamification}. Introducing game mechanics into the context of the solution will make menial tasks such as manually grading the sentiment of each post more fun, rewarding and desirable, as most people enjoy participating in some form of game~\parencite{hagglund2012taking}. Consequently, the group think of ways to implement gamification into the solution.

Furthermore, the group must consider further leveraging the educational component of the solution. It should be expected that not many parties are fluent with the concept of the right-to-forgotten, hence every attempt must be made to inform them of what options are available to them within this legal context.
