\section{Ideas}

\subsection{Overview}

\begin{figure}
  \centering
  \begin{minipage}{7cm}
    \centering
    \includegraphics[width=7cm]{inc/breadcrumb_logo.png}
    \caption{Breadcrumb Logo}
    \label{fig:breadcrumbLogo}
  \end{minipage}
\end{figure}

Breadcrumb, the title of group's proposed solution, is an impression management tool that helps individuals and corporates positively manage and optimise their online information. The name comes from the story of Hansel and Gretel, who left a trail of breadcrumbs as they walked through the forest so they could trace their way home. In the same way, the group noted individuals and corporates often leave a digital footprint that makes it easy for others to develop an opinion about their reputation from what they see online. Unlike footprints left in the sand at the beach, an online ``breadcrumb trail'' often sticks around long after the tide has gone, and can have negative implications for all involved.

\subsection{Business Model}

Breadcrumb will come in the form of three-tiered structured offering, which will provide clients with a customised level of service at different price points in order to appeal to a wider segment.

\subsubsection{Tier 1: Personal}

The Personal offering is geared towards individuals wanting to monitor and control their online information -- for instance, prior to making a job application. In this case, when Breadcrumb is run for the first time, the user will be asked to connect all their social media accounts (i.e.~Facebook, Instagram and Twitter) and provide their personal information. Breadcrumb then scans each social media account for content it initially considers to be significantly positive or negative and collates a primary list. Further to this, it trawls the internet to find material that it initially considers to be significantly positive or negative and collates a secondary list. As Breadcrumb is an impression management tool, it is important for it to understand the entirety of someone's online profile, so that it can recommend ways in which the user can improve their digital footprint. Once both lists are collated, they are combined and presented to the user as a ``graphical deck of cards''. Each card contains information on the content Breadcrumb identified as significantly positive or negative. At this point, the user can swipe through each card and decide whether they personally deem the content to be positive or negative. Consequently, Breadcrumb is trained to understand what the user wants their online profile to look like, so that in future scans, it will get better at flagging relevant material. Over time, this increases Breadcrumb's value or stickiness, making it rather difficult for users to switch.

As the group expects the Personal offering to be used infrequently, it will employ a freemium pricing strategy for this portion of the service. As this service is new to market, offering a freemium version to individual users can be a quick way to drive trial usage and generate feedback before scaling up because of the low barrier to entry.

In order to convert free users into paying users, Breadcrumb will only display a limited number of results but charge users a small fee in order to unlock the rest. Moreover, Breadcrumb will regularly provide the user with recommendations on how to improve their online profile, which can be unlocked for another small fee. These recommendations will be derived from what Breadcrumb understands about the user and may involve ideas around content production. All of these features will be presented to the user in a gamified manner, often with the promise of ``rewards'' that motivate the user to keep returning due to their perceived fairness of the transaction. Moreover, such a system will also give the user a sense of control over their continued investment.

\begin{figure}
  \centering
  \begin{minipage}{4cm}
    \centering
    \includegraphics[width=4cm]{inc/flip_ui.jpg}
    \caption{Card User Interface}
    \label{fig:flipUI}
  \end{minipage}
\end{figure}

\subsubsection{Tier 2: Premier}

The Premier offering is geared towards individuals wanting to understand and shape public opinion, create compelling narratives and develop initiatives that persuade people to act -- for instance, a agent aiming to maximise the market value of their client (in this case, a football player) would want to have them conveyed in the media as a skilful player in order to inspire a bid from a particular football club.

The Premier offering contains the same features as the Personal offering, but also provides users the opportunity to review multiple online profiles at once and derive a sentiment score which measures the emotional tone of the information found, along with current public perception. This allows for a variety of interesting use cases -- for example, a spin-doctor coordinating a political campaign in the build-up to a presidential election can compare the sentiment score between two competing candidates in order to better understand how the public is reacting to their client’s recent activities.

Further to this, Breadcrumb can provide the spin-doctor or agent with a unique, in-depth impression management strategy that can help improve their client’s sentiment score and overall public image. The group envisions that the Premier offering could also be used by corporates seeking to maintain the online reputation of their executives, and by celebrities in order to identify instances of copyright infringement, wherein their work or likeness have been used without prior permission. 

As the group expects the Premier offering to be regularly used, it will employ a subscription pricing strategy for this portion of the service. Users have the chance to use Breadcrumb free for 30 days in order to determine whether it is suitable for them, but beyond that they will need to pay a monthly fee.

\subsubsection{Tier 3: Enterprise}

The Enterprise offering is geared towards small, medium and large corporates wanting to aggregate relevant content, analyse the sentiment of each mention and derive information about the individuals hosting discussions about the brand. Further to this, Breadcrumb will provide corporates with tailored, brand amplification strategies. The group also envisions that the Enterprise offering could also be used by corporates screening job applicants, subsequently providing human resource departments with insights into whether or not an individual is a fit for the organisation based on their online profile. 

As the group expects the Enterprise offering to be regularly used, it will employ a subscription pricing strategy for this portion of the service identical to the Premier offering, as it builds upon that package.

\subsection{Use Case Diagram}

\begin{figure}
  \centering
  \begin{minipage}{14cm}
    \centering
    \includegraphics[width=14cm]{inc/use_case_diagram.png}
    \caption{Use Case Diagram}
    \label{fig:useCaseDiagram}
  \end{minipage}
\end{figure}

\clearpage
\begin{landscape}

\subsection{Lean Canvas}

The group considers Maurya's Lean Canvas to be the best choice in modelling the business case. It is part of a methodology called "`lean start-up"', which favours iterative design over “big design up front” development, and does not begin with a business plan, but rather in search for a business model. Given the close overlap between business and technology in the assigned context, it is very important to identify any synergy potential early on.

\begin{figure}
  \centering
  \begin{minipage}{160mm}
    \centering
    \includegraphics[width=160mm]{inc/lean_canvas.png}
    \caption{Lean Canvas}
    \label{fig:lean_canvas}
  \end{minipage}
\end{figure}

\end{landscape}


